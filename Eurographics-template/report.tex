% ---------------------------------------------------------------------------
% Author guideline and sample document for EG publication using LaTeX2e input
% D.Fellner, v2, June 1, 2017

\documentclass{egpubl}

% --- for  Annual CONFERENCE
% \ConferenceSubmission % uncomment for Conference submission
% \ConferencePaper      % uncomment for (final) Conference Paper
% \STAR                 % uncomment for STAR contribution
% \Tutorial             % uncomment for Tutorial contribution
% \ShortPresentation    % uncomment for (final) Short Conference Presentation
%
% --- for  CGF Journal
\JournalSubmission    % uncomment for submission to Computer Graphics Forum
%  \JournalPaper         % uncomment for final version of Journal Paper (NOTE: this won't have page numbers)
%
% --- for  CGF Journal: special issue
% \SpecialIssueSubmission    % uncomment for submission to Computer Graphics Forum, special issue
% \SpecialIssuePaper         % uncomment for final version of Journal Paper, special issue
%
% --- for  EG Workshop Proceedings
% \WsSubmission    % uncomment for submission to EG Workshop
% \WsPaper         % uncomment for final version of EG Workshop contribution
%
 \electronicVersion % can be used both for the printed and electronic version

% !! *please* don't change anything above
% !! unless you REALLY know what you are doing
% ------------------------------------------------------------------------

% for including postscript figures
% mind: package option 'draft' will replace PS figure by a filname within a frame
\ifpdf \usepackage[pdftex]{graphicx} \pdfcompresslevel=9
\else \usepackage[dvips]{graphicx} \fi

\PrintedOrElectronic

% prepare for electronic version of your document
\usepackage{t1enc,dfadobe}

\usepackage{egweblnk}
\usepackage{cite}

% For backwards compatibility to old LaTeX type font selection.
% Uncomment if your document adheres to LaTeX2e recommendations.
% \let\rm=\rmfamily    \let\sf=\sffamily    \let\tt=\ttfamily
% \let\it=\itshape     \let\sl=\slshape     \let\sc=\scshape
% \let\bf=\bfseries

% end of prologue

% \input{EGauthorGuidelines-body.inc}

\title[Polygonal Model Compression]%
      {Polygonal Model Compression with Graph Symmetries}

% for anonymous conference submission please enter your SUBMISSION ID
% instead of the author's name (and leave the affiliation blank) !!
\author[Andrej Jočić]
{\parbox{\textwidth}{\centering% D.\,W. Fellner\thanks{Chairman Eurographics Publications Board}$^{1,2}$
        Andrej Jočić$^{1}$ 
%        S. Spencer$^2$\thanks{Chairman Siggraph Publications Board}
        }
        \\
% For Computer Graphics Forum: Please use the abbreviation of your first name.
{\parbox{\textwidth}{\centering %$^1$TU Darmstadt \& Fraunhofer IGD, Germany\\
         $^1$University of Ljubljana, Faculty of Computer and Information Science, Slovenia
%        $^2$ Another Department to illustrate the use in papers from authors
%             with different affiliations
       }
}
}
% ------------------------------------------------------------------------

% if the Editors-in-Chief have given you the data, you may uncomment
% the following five lines and insert it here
%
% \volume{36}   % the volume in which the issue will be published;
% \issue{1}     % the issue number of the publication
% \pStartPage{1}      % set starting page


%-------------------------------------------------------------------------
\begin{document}

% uncomment for using teaser
% \teaser{
%  \includegraphics[width=\linewidth]{eg_new}
%  \centering
%   \caption{New EG Logo}
% \label{fig:teaser}
%}

\maketitle
%-------------------------------------------------------------------------
\begin{abstract}

TODO the abstract
%-------------------------------------------------------------------------
%  ACM CCS 1998
%  (see http://www.acm.org/about/class/1998)
% \begin{classification} % according to http:http://www.acm.org/about/class/1998
% \CCScat{Computer Graphics}{I.3.3}{Picture/Image Generation}{Line and curve generation}
% \end{classification}
%-------------------------------------------------------------------------
%  ACM CCS 2012
   (see http://www.acm.org/about/class/class/2012)
%The tool at \url{http://dl.acm.org/ccs.cfm} can be used to generate
\begin{CCSXML}
    <ccs2012>
    <concept>
    <concept_id>10010147.10010371.10010396.10010397</concept_id>
    <concept_desc>Computing methodologies~Mesh models</concept_desc>
    <concept_significance>500</concept_significance>
    </concept>
    <concept>
    <concept_id>10010147.10010371.10010387.10010394</concept_id>
    <concept_desc>Computing methodologies~Graphics file formats</concept_desc>
    <concept_significance>300</concept_significance>
    </concept>
    <concept>
    <concept_id>10003752.10003809.10010031.10002975</concept_id>
    <concept_desc>Theory of computation~Data compression</concept_desc>
    <concept_significance>100</concept_significance>
    </concept>
</ccs2012>
\end{CCSXML}

\ccsdesc[500]{Computing methodologies~Mesh models}
\ccsdesc[300]{Computing methodologies~Graphics file formats}
\ccsdesc[100]{Theory of computation~Data compression}


\printccsdesc   
\end{abstract}  
%-------------------------------------------------------------------------
\section{Introduction}

Polygonal models are widely used for representing 3D objects in computer graphics. With increased rendering system capabilities and the desire for greater precision, the size of these models has been increasing. To cut down transmission and static storage costs, specialized compression schemes have been developed.
In previously established \cite{meshCompressionSurvey} terms, we will be focusing on lossless single-rate global compression of static polygonal models. This means that we will be considering fixed (non-evolving) data, which will be decompressed it all at once (without progressive transmission at different levels of detail) and in its entirety (not focusing on specific regions of the model as requested by the user).

The information we usually store is the model's geometry (positions of vertices), connectivity / topology (incidence relations among elements), and optionally also attributes (normals, colours, etc.).
Typically the connectivity information takes up the most space (around twice as much as the goemetry information in a triangle mesh homeomorphic to a sphere \cite{rossignac1999edgebreaker}), so it is the most important to compress.
Many compression methods focus exclusively on manifold triangle meshes (or another constrained structure) to get the best possible results.  
In this paper we will consider the case of arbitrary polygonal models. Since the connectivity relations can be represented as an undirected graph, we can reduce the problem to that of graph compression.
Specifically, we will be exploiting symmetries in the graph, which can often occur in the graphical domain.

\subsection{Related Work}

% TODO: slap some survey papers in here (graph compression, mesh compression)

% TODO: list connectivity comp. rates from survey table 1
this an that paper hasdealt with symmetry-based mesh compression, but to our knowledge it has not been done with automorphisms.

\section{Methods}

Čibej and Mihelič \cite{cibej2021automorphisms} have developed a method general graph compression based on automorphisms. They define
\textit{symmetry-compressible} graphs $G$
as ones that can be more succinctly represented with a ``residual'' graph $G^\pi$ and an automorhpism (connectivity-preserving vertex permutation) $\pi$ of $G$.
Formally, $G \in \mathcal{S}\mathcal{C}$ if $\exists \pi \in Aut(G)$ such that $|G^\pi| + |\pi| < |G|$, where $|\cdot|$ denotes a measure of representation size (e.g. number of edges). 

% TODO: define near symmetry-compressible graphs


The main computational issue is finding a (compressive) automorphism. The paper presents two approaches: graphlet search and bipartite completion.
As Theorem TODO shows, we can just search for SC subgraphs.

\subsection{Graphlet Search}

\subsection{Bipartite Completion}

Though a star graph is a common occurence in the graphical domain, it is unfortunately not symmetry-compressible (measuring size by number of edges) (??) do we only need NSC?


\section{Results}


\section{Discussion}

\subsection{Conclusions}

Please direct any questions to the production editor in charge of
these proceedings.

%-------------------------------------------------------------------------

%\bibliographystyle{eg-alpha}
\bibliographystyle{eg-alpha-doi}

\bibliography{references}

\end{document}
